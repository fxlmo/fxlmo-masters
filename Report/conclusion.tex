\chapter{Conclusion}
\label{chap:conclusion}

\noindent
The concluding chapter of a dissertation is often underutilised because it 
is too often left too close to the deadline: it is important to allocate
enough attention to it.  Ideally, the chapter will consist of three parts:

\begin{enumerate}
\item (Re)summarise the main contributions and achievements, in essence
      summing up the content.
\item Clearly state the current project status (e.g., ``X is working, Y 
      is not'') and evaluate what has been achieved with respect to the 
      initial aims and objectives (e.g., ``I completed aim X outlined 
      previously, the evidence for this is within Chapter Y'').  There 
      is no problem including aims which were not completed, but it is 
      important to evaluate and/or justify why this is the case.
\item Outline any open problems or future plans.  Rather than treat this
      only as an exercise in what you {\em could} have done given more 
      time, try to focus on any unexplored options or interesting outcomes
      (e.g., ``my experiment for X gave counter-intuitive results, this 
      could be because Y and would form an interesting area for further 
      study'' or ``users found feature Z of my software difficult to use,
      which is obvious in hindsight but not during at design stage; to 
      resolve this, I could clearly apply the technique of Smith [7]'').
\end{enumerate}

% =============================================================================

% Finally, after the main matter, the back matter is specified.  This is
% typically populated with just the bibliography.  LaTeX deals with these
% in one of two ways, namely
%
% - inline, which roughly means the author specifies entries using the 
%   \bibitem macro and typesets them manually, or
% - using BiBTeX, which means entries are contained in a separate file
%   (which is essentially a databased) then inported; this is the 
%   approach used below, with the databased being dissertation.bib.
%
% Either way, the each entry has a key (or identifier) which can be used
% in the main matter to cite it, e.g., \cite{X}, \cite[Chapter 2}{Y}.
%
% We would recommend using BiBTeX, since it guarantees a consistent referencing style 
% and since many sites (such as dblp) provide references in BiBTeX format. 
% However, note that by default, BiBTeX will ignore capital letters in article titles 
% to ensure consistency of style. This can lead to e.g. "NP-completeness" becoming
% "np-completeness". To avoid this, make sure any capital letters you want to preserve
% are enclosed in braces in the .bib, e.g. "{NP}-completeness".